\documentclass{article}
\usepackage{listings}
\usepackage{graphicx}
\usepackage{float}
\usepackage[a4paper, total={6in, 9in}]{geometry}
\graphicspath{ {./} }
\usepackage{Sweave}
\begin{document}
\input{doc-concordance}


\title{Factorization methods analysis}
\author{Bereczki Norbert Cristian}
\maketitle

\section{Methods}
\subsection{Primitive algorithm}
For i in the range {2,...,n/2} try to see if i|n. If yes then break. Elsewhere continue on. 
\subsection{Pollard's p algorithm}
Let \[x_0 = 2 .\]
For j = 1, 2, . . . compute the sequence:
\[x_j = f (x_j - 1)\pmod{n} \]
and \[d = (|x_{2j} - x_j|, n).\]

If 1 < d < n, then STOP and d is a non-trivial factor of n.
If d = n, then STOP and FAILURE. In this case, one can
repeat the algorithm with a different x0 or f .
Else, continue with the next value of j.

\section{Runtime analysis}
\begin{center}
 \begin{tabular}{||c c c||} 
 \hline
 Input & Primitive Algo & Pollard's p \\ [0.5ex] 
 \hline\hline
 911352783367 & 0 & 100 \\ 
 \hline
 163398410325 & 0 & 0 \\
 \hline
 623926552581 & 0 & 0 \\
 \hline
 314047195607 & 0 & 0 \\
 \hline
 849257520909 & 0 & 0 \\ [1ex] 
 \hline
 806442382101 & 0 & 0 \\ [1ex] 
 \hline
 125200496397 & 0 & 0 \\ [1ex] 
 \hline
 130417715505 & 0 & 0 \\ [1ex] 
 \hline
 1000119529 & 0 & 2 \\ [1ex] 
 \hline
 833837197611 & 0 & 0 \\ [1ex] 
 \hline
 10000001400000049 & 1074 & doesn't stop \\ [1ex] 
 \hline
\end{tabular}
\end{center}












\end{document}
