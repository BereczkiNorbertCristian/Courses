\documentclass{article}

\usepackage{listings}
\usepackage{graphicx}
\usepackage{float}
\usepackage[a4paper, total={6in, 9in}]{geometry}
\usepackage{mathtools}

\usepackage{Sweave}
\begin{document}
\Sconcordance{concordance:Doc.tex:Doc.Rnw:%
1 8 1 1 0 30 1}


\title{GCD Algorithms}
\author{Bereczki Norbert Cristian}
\maketitle

\abstract{
  A composite natural number \textit{n} is called a \textit{Carmichael number} if $ b^{n-1} = 1 (mod n)$ (1) holds $\forall b \in Z$ with (b,n) = 1 .
}

\section{Problem}
Implement an algorithm for determining all Carmichael numbers less than a given bound.

\section{Algorithm used}

\subsection{Brute force}

For each x from 3 to Upper Bound check if it is composite, then iterate (with i) through 2 to x and check if gcd(x,i)=1 and that (1) holds.

\subsection{Smart way}

Used the following properties:
If n is square free (that is, it is not divisible by the square of
any prime), then n is a Carmichael number $\iff$ p-1|n-1 $\forall p$ prime that p|n. 

First we precompute the Sieve of Eratosthenes up to the given upper bound. 
Given an upper bound iterate through all values x from 3 to the upper bound. If x is even we skip over it. We check if any of it's divisors is prime, and if it is we check that each prime divisor obeys: p-1 | n-1. Then we check that the number is squarefree by iterating through all prime numbers in the sieve until [$\sqrt{x}$]. If all is true until now then the x is a Carmichael number.


\end{document}
